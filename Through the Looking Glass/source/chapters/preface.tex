\chapter*{Preface}

As the chess-problem, given on a previous page, has puzzled some of my readers, it may be well to explain that it is correctly worked out, so far as the \textit{moves} are concerned. The \textit{alternation} of Red and White is perhaps not so strictly observed as it might be, and the »castling« of the three Queens is merely a way of saying that they entered the palace; but the »check« of the White King at move 6, the capture of the Red Knight at move 7, and the final »check-mate« of the Red King, will be found, by any one who will take the trouble to set the pieces and play the moves as directed, to be strictly in accordance with the laws of the game.

The new words, in the poem »Jabberwocky« (see page \pageref{jabberwocky}), have given rise to some differences of opinion as to their pronunciation: so it may be well to give instructions on \textit{that} point also. Pronounce »slithy« as if it were to the words »sly, the«; make the »g« \textit{hard} in »gyre« and »gimble«; and pronounce »rath« to rhyme with »bath.«

~\\
~\\
Christmas, 1896